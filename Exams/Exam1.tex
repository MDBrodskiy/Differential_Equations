%%%%%%%%%%%%%%%%%%%%%%%%%%%%%%%%%%%%%%%%%%%%%%%%%%%%%%%%%%%%%%%%%%%%%%%%%%%%%%%%%%%%%%%%%%%%%%%%%%%%%%%%%%%%%%%%%%%%%%%%%%%%%%%%%%%%%%%%%%%%%%%%%%%%%%%%%%%%%%%%%%%%%%%%%%%%%%%%%%%%%%%%%%%%
% Written By Michael Brodskiy
% Class: Differential Equations (MATH-294)
% Professor: M. Shah
%%%%%%%%%%%%%%%%%%%%%%%%%%%%%%%%%%%%%%%%%%%%%%%%%%%%%%%%%%%%%%%%%%%%%%%%%%%%%%%%%%%%%%%%%%%%%%%%%%%%%%%%%%%%%%%%%%%%%%%%%%%%%%%%%%%%%%%%%%%%%%%%%%%%%%%%%%%%%%%%%%%%%%%%%%%%%%%%%%%%%%%%%%%%

\documentclass[12pt]{article} 
\usepackage{alphalph}
\usepackage[utf8]{inputenc}
\usepackage[russian,english]{babel}
\usepackage{titling}
\usepackage{amsmath}
\usepackage{graphicx}
\usepackage{enumitem}
\usepackage{amssymb}
\usepackage[super]{nth}
\usepackage{everysel}
\usepackage{ragged2e}
\usepackage{geometry}
\usepackage{fancyhdr}
\usepackage{cancel}
\usepackage{siunitx}
\geometry{top=1.0in,bottom=1.0in,left=1.0in,right=1.0in}
\newcommand{\subtitle}[1]{%
  \posttitle{%
    \par\end{center}
    \begin{center}\large#1\end{center}
    \vskip0.5em}%

}
\usepackage{hyperref}
\hypersetup{
colorlinks=true,
linkcolor=blue,
filecolor=magenta,      
urlcolor=blue,
citecolor=blue,
}

\urlstyle{same}


\title{Differential Equations $-$ Exam One}
\date{\today}
\author{Michael Brodskiy\\ \small Professor: Meetal Shah}

% Mathematical Operations:

% Sum: $$\sum_{n=a}^{b} f(x) $$
% Integral: $$\int_{lower}^{upper} f(x) dx$$
% Limit: $$\lim_{x\to\infty} f(x)$$

\begin{document}

\maketitle

\begin{enumerate}

  \item \eqref{1}

  \item \eqref{2}

  \item \eqref{3}

  \item \eqref{4} 

  \item \eqref{5}

  \item \eqref{6}

  \item \eqref{7}

  \item \eqref{8}

  \item \eqref{9}

  \item \eqref{10}

  \item \eqref{11}

\end{enumerate}

\hline
\begin{equation}
  \begin{split}
    \frac{d^2R}{dt^2} & = -\frac{k}{R^2} \\
    \text{The differential equation is NON-linear}\\ 
    \text{The differential equation is an ODE}\\
  \end{split}
  \label{1}
\end{equation}
\hline

\begin{equation}
  \begin{split}
    \int \frac{1}{y^{.5}}\,dy=\int x^{.5}\,dx\\
    .5y^{.5}=\frac{2}{3}x^{1.5}+C \\
    \text{This function has a solution where: } x > 0, y > 0
  \end{split}
  \label{2}
\end{equation}

\hline
\begin{equation}
  \begin{split}
    \frac{dx}{dt} & = kxn-kx^2 \\
  \end{split}
  \label{3}
\end{equation}

\hline
\begin{equation}
  \begin{split}
    10+3y-y^2 & = 0 \\
    (5-y)(y+2) & = 0 \\
    y & = 5, -2 \\
    y = 5 \text{ is stable } & y = -2 \text{ is unstable } \\
  \end{split}
  \label{4}
\end{equation}

\hline
\begin{equation}
  \begin{split}
    \int(2y-2)\,dy & = \int 3x^2+4x+2\,dx \\
    y^2-2y & = x^3+2x^2+2x+C \\
    C = 8-5 & = 3 \\
    y^2-2y & = x^3+2x^2+2x+3 \\
  \text{This function is defined in: } & (-\infty,\infty)\\
  \end{split}
  \label{5}
\end{equation}
\hline

\begin{equation}
  \begin{split}
    \frac{dx}{dy}-\frac{x}{y}=2y^2 \\
    I = e^{-\int\frac{1}{y}}=\frac{1}{y}\\
    \int\left( \frac{x}{y} \right)'\,dx=\int\left( 2y \right)\,dy\\
    \frac{x}{y}=y^2+C\\
    C=-\frac{124}{5} \\
    \frac{x}{y}=y^2-\frac{124}{5} \\
    \text{This function is defined in: } y\neq0 
  \end{split}
  \label{6}
\end{equation}
\hline


\hline
\begin{equation}
  \begin{split}
    (xy+y^2+y)\,dx+(x+2y)\,dy=0\\
    \text{The function is exact}\\
    \int(x+2y)\,dy=xy+y^2+h(x)\\
    \cancel{y}+h'(x)=xy+y^2+\cancel{y}\\
    h(x)=\frac{x^2y}{2}+xy^2\\
    xy+y^2+xy^2+\frac{x^2y}{2}=C\\
  \end{split}
  \label{7}
\end{equation}
\hline

\begin{equation}
  \begin{split}
    (x^2+2y^2)\frac{dx}{dy}=xy, x=vy \\
    v+y\frac{dv}{dy}= \frac{v}{v^2+2} \\
    y\frac{dv}{dy}=\frac{-v^3-v}{v^2+2} \\
    \frac{v^2+2}{-v^3-v}\,dv=\frac{1}{y}\,dy\\
    -\int\frac{v}{v^2+1}+\frac{2}{v^3+v}\,dv=\int\frac{1}{y}\\
    \frac{1}{2}\ln(|(\frac{x}{y})^2+1|)-2\ln(|(\frac{x}{y})|)=\ln(|y|)+C\\
    C=\frac{1}{2}\ln(2)\\
    \frac{1}{2}\ln(|(\frac{x}{y})^2+1|)-2\ln(|(\frac{x}{y})|)=\ln(|y|)+\frac{1}{2}\ln(2)\\
  \end{split}
  \label{8}
\end{equation}


\hline
\begin{equation}
  \begin{split}
    e(x,y)=y+.2xy\\
    e(1,1)=1.2\\
    e(1.1,1.2)=1.464\\
    e(1.2,1.464)=1.81536\\
    e(1.3,1.81536)=2.28735\\
    e(1.4,2.28735)=2.92781\\
    y(1.5)\approx2.92781
  \end{split}
  \label{9}
\end{equation}

\hline
\begin{equation}
  \begin{split}
    \frac{dx_1}{dt}=\frac{x_2}{50}-\frac{3x_1}{50}\\
    \frac{dx_2}{dt}=\frac{3x_1}{50}-\frac{7x_2}{100}+\frac{x_3}{100}\\
    \frac{dx_3}{dt}=\frac{x_2}{20}-\frac{x_3}{20}\\
  \end{split}
  \label{10}
\end{equation}
\hline

\begin{equation}
  \begin{split}
  y'+\left( \frac{1+x}{x} \right)y=\frac{\sin(2x)}{e^x \cdot x}\\
  I=e^{ln(|x|)+x}=xe^x\\
  \int\left( xe^x y \right)'\,dy=\int \sin(2x)\,dx\\
  xe^x y=\frac{-\cos(2x)}{2}\\
  \text{The biggest interval is: } x = (0,\infty)
  \end{split}
  \label{11}
\end{equation}
\hline

\end{document}

