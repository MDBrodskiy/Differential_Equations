%%%%%%%%%%%%%%%%%%%%%%%%%%%%%%%%%%%%%%%%%%%%%%%%%%%%%%%%%%%%%%%%%%%%%%%%%%%%%%%%%%%%%%%%%%%%%%%%%%%%%%%%%%%%%%%%%%%%%%%%%%%%%%%%%%%%%%%%%%%%%%%%%%%%%%%%%%%%%%%%%%%%%%%%%%%%%%%%%%%%%%%%%%%%
% Written By Michael Brodskiy
% Class: Differential Equations (MATH-294)
% Professor: M. Shah
%%%%%%%%%%%%%%%%%%%%%%%%%%%%%%%%%%%%%%%%%%%%%%%%%%%%%%%%%%%%%%%%%%%%%%%%%%%%%%%%%%%%%%%%%%%%%%%%%%%%%%%%%%%%%%%%%%%%%%%%%%%%%%%%%%%%%%%%%%%%%%%%%%%%%%%%%%%%%%%%%%%%%%%%%%%%%%%%%%%%%%%%%%%%

\documentclass[12pt]{article} 
\usepackage{alphalph}
\usepackage[utf8]{inputenc}
\usepackage[russian,english]{babel}
\usepackage{titling}
\usepackage{amsmath}
\usepackage{graphicx}
\usepackage{enumitem}
\usepackage{amssymb}
\usepackage[super]{nth}
\usepackage{everysel}
\usepackage{ragged2e}
\usepackage{geometry}
\usepackage{fancyhdr}
\usepackage{cancel}
\usepackage{siunitx}
\geometry{top=1.0in,bottom=1.0in,left=1.0in,right=1.0in}
\newcommand{\subtitle}[1]{%
  \posttitle{%
    \par\end{center}
    \begin{center}\large#1\end{center}
    \vskip0.5em}%

}
\usepackage{hyperref}
\hypersetup{
colorlinks=true,
linkcolor=blue,
filecolor=magenta,      
urlcolor=blue,
citecolor=blue,
}

\urlstyle{same}


\title{Linear Equations}
\date{\today}
\author{Michael Brodskiy\\ \small Professor: Meetal Shah}

% Mathematical Operations:

% Sum: $$\sum_{n=a}^{b} f(x) $$
% Integral: $$\int_{lower}^{upper} f(x) dx$$
% Limit: $$\lim_{x\to\infty} f(x)$$

\begin{document}

\maketitle

\begin{itemize}

  \item A first-order linear differential equation takes the form:

    $$a_1(x)\frac{dy}{dx}+a_0(x)y=g(x)$$

  \item The standard form (where $P(x)=\frac{a_0(x)}{a_1(x)}$ and $f(x)=g(x)$) is as follows:

      $$\frac{dy}{dx}+P(x)y=f(x)$$

    \item The solution of such differential equations may be obtained by using the product rule:

      $$\frac{d}{dx}[\mu(x)y]=\mu\frac{dy}{dx}+\frac{d\mu}{dx}y=\mu\frac{dy}{dx}+\mu Py$$
      \begin{center} This is proven true when: \end{center}

    $$\frac{d\mu}{dx}=\mu P,\text{ or } \mu(x)=c_2e^{\int P(x)\,dx}$$

    \item This value is known as the integrating factor, $I$

    \item Solving a Linear First-Order Equation

      \begin{enumerate}

        \item First, rearrange the equation into standard form

        \item Find the integrating factor by locating $P(x)$

        \item Multiply both sides by the integrating factor (and add constant $c$).

        \item Then, solve for $y$

      \end{enumerate}

    \item Values of $x$ where $a_1(x)=0$ are known as singular points ($a_1(x)$ is the coefficient of $\frac{dy}{dx}$)

    \item A transient term is one that approaches zero as $x$ approaches infinity (ex. $e^{-x}$)

    \item When either $P(x)$ or $f(x)$ is a piecewise-defined function, the equation is said to be a piecewise-linear differential equation

    \item Two important functions defined by integrals exist. These are the error function and the complimentary error function, respectively:

      $$erf(x)=\frac{2}{\sqrt{\pi}}\int_0^x e^{-t^2}\,dt\,\,\,\,\,\,\,\, erfc(x)=\frac{2}{\sqrt{\pi}}\int_x^{\infty} e^{-t^2}\,dt$$
      
    \item These two functions are related by an identity:

      $$erf(x)+erfc(x)=1$$

\end{itemize}

\end{document}

