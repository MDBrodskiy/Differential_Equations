%%%%%%%%%%%%%%%%%%%%%%%%%%%%%%%%%%%%%%%%%%%%%%%%%%%%%%%%%%%%%%%%%%%%%%%%%%%%%%%%%%%%%%%%%%%%%%%%%%%%%%%%%%%%%%%%%%%%%%%%%%%%%%%%%%%%%%%%%%%%%%%%%%%%%%%%%%%%%%%%%%%%%%%%%%%%%%%%%%%%%%%%%%%%
% Written By Michael Brodskiy
% Class: Differential Equations (MATH-294)
% Professor: M. Shah
%%%%%%%%%%%%%%%%%%%%%%%%%%%%%%%%%%%%%%%%%%%%%%%%%%%%%%%%%%%%%%%%%%%%%%%%%%%%%%%%%%%%%%%%%%%%%%%%%%%%%%%%%%%%%%%%%%%%%%%%%%%%%%%%%%%%%%%%%%%%%%%%%%%%%%%%%%%%%%%%%%%%%%%%%%%%%%%%%%%%%%%%%%%%

\documentclass[12pt]{article} 
\usepackage{alphalph}
\usepackage[utf8]{inputenc}
\usepackage[russian,english]{babel}
\usepackage{titling}
\usepackage{amsmath}
\usepackage{graphicx}
\usepackage{enumitem}
\usepackage{amssymb}
\usepackage[super]{nth}
\usepackage{everysel}
\usepackage{ragged2e}
\usepackage{geometry}
\usepackage{fancyhdr}
\usepackage{cancel}
\usepackage{siunitx}
\geometry{top=1.0in,bottom=1.0in,left=1.0in,right=1.0in}
\newcommand{\subtitle}[1]{%
  \posttitle{%
    \par\end{center}
    \begin{center}\large#1\end{center}
    \vskip0.5em}%

}
\usepackage{hyperref}
\hypersetup{
colorlinks=true,
linkcolor=blue,
filecolor=magenta,      
urlcolor=blue,
citecolor=blue,
}

\urlstyle{same}


\title{Solutions About Ordinary Points}
\date{\today}
\author{Michael Brodskiy\\ \small Professor: Meetal Shah}

% Mathematical Operations:

% Sum: $$\sum_{n=a}^{b} f(x) $$
% Integral: $$\int_{lower}^{upper} f(x) dx$$
% Limit: $$\lim_{x\to\infty} f(x)$$

\begin{document}

\maketitle

\begin{itemize}

  \item A point $x=x_0$ is said to be an ordinary point of the differential of the differential equation $a_2(x)y''+a_1(x)y'+a_0(x)y=0$ if both coefficients $P(x)$ and $Q(x)$ in the standard form $y''+P(x)y'+Q(x)y=0$ are analytic (can be expressed as a series) at $x_0$. A point that is not an ordinary point is said to be a singular point of the DE. \eqref{1}

    \begin{equation}
      \begin{split}
        P(x)=\frac{a_1(x)}{a_2(x)} \text{ and }& Q(x)=\frac{a_0(x)}{a_2(x)}
      \end{split}
      \label{1}
    \end{equation}

  \item A point is ordinary as long as $a_2(x)\neq0$ at $x_0$\footnote{NOTE: singular points do not have to real. For example, in $(x^2+1)y''+xy'-y=0$, $x^2+1$ has roots $\pm1$}

  \item If $x=x_0$ is an ordinary point of the differential equation, we can always find two linearly independent solutions in the form of a power series centered at $x_0$, that is \eqref{2}. A power series solution converges at least on some interval defined by $|x-x_0|<R$, where $R$ is the distance from $x_0$ to the closest singular point.

    \begin{equation}
      y=\sum_{n=0}^{\infty} c_n(x-x_0)^n
      \label{2}
    \end{equation}

  \item A solution about the ordinary point $x_0$ is a solution of form \eqref{2}

\end{itemize}

\end{document}

