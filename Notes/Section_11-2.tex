%%%%%%%%%%%%%%%%%%%%%%%%%%%%%%%%%%%%%%%%%%%%%%%%%%%%%%%%%%%%%%%%%%%%%%%%%%%%%%%%%%%%%%%%%%%%%%%%%%%%%%%%%%%%%%%%%%%%%%%%%%%%%%%%%%%%%%%%%%%%%%%%%%%%%%%%%%%%%%%%%%%%%%%%%%%%%%%%%%%%%%%%%%%%
% Written By Michael Brodskiy
% Class: Differential Equations (MATH-294)
% Professor: M. Shah
%%%%%%%%%%%%%%%%%%%%%%%%%%%%%%%%%%%%%%%%%%%%%%%%%%%%%%%%%%%%%%%%%%%%%%%%%%%%%%%%%%%%%%%%%%%%%%%%%%%%%%%%%%%%%%%%%%%%%%%%%%%%%%%%%%%%%%%%%%%%%%%%%%%%%%%%%%%%%%%%%%%%%%%%%%%%%%%%%%%%%%%%%%%%

\documentclass[12pt]{article} 
\usepackage{alphalph}
\usepackage[utf8]{inputenc}
\usepackage[russian,english]{babel}
\usepackage{titling}
\usepackage{amsmath}
\usepackage{graphicx}
\usepackage{enumitem}
\usepackage{amssymb}
\usepackage[super]{nth}
\usepackage{everysel}
\usepackage{ragged2e}
\usepackage{geometry}
\usepackage{fancyhdr}
\usepackage{cancel}
\usepackage{siunitx}
\geometry{top=1.0in,bottom=1.0in,left=1.0in,right=1.0in}
\newcommand{\subtitle}[1]{%
  \posttitle{%
    \par\end{center}
    \begin{center}\large#1\end{center}
    \vskip0.5em}%

}
\usepackage{hyperref}
\hypersetup{
colorlinks=true,
linkcolor=blue,
filecolor=magenta,      
urlcolor=blue,
citecolor=blue,
}

\urlstyle{same}


\title{Fourier Series}
\date{\today}
\author{Michael Brodskiy\\ \small Professor: Meetal Shah}

% Mathematical Operations:

% Sum: $$\sum_{n=a}^{b} f(x) $$
% Integral: $$\int_{lower}^{upper} f(x) dx$$
% Limit: $$\lim_{x\to\infty} f(x)$$

\begin{document}

\maketitle

\begin{itemize}

  \item For any function defined on the interval $(-p,p)$, an expression can be obtained that looks like \eqref{1}

    \begin{equation}
    f(x)=\frac{a_0}{2}+\sum_{n=1}^{\infty}\left( a_n\cos\frac{n\pi}{p}x+b_n\sin\frac{n\pi}{p}x \right)
      \label{1}
    \end{equation}

  \item The components of a Fourier Series may be found by using formulas \eqref{2}, \eqref{3}, and \eqref{4}

    \begin{equation}
      a_0=\frac{1}{p}\int_{-p}^pf(x)\,dx
      \label{2}
    \end{equation}

    \begin{equation}
      a_n=\frac{1}{p}\int_{-p}^pf(x)\cos\frac{n\pi x}{p}\,dx
      \label{3}
    \end{equation}

    \begin{equation}
      b_n=\frac{1}{p}\int_{-p}^pf(x)\sin\frac{n\pi x}{p}\,dx
      \label{4}
    \end{equation}

  \item The series converges at the point defined by \eqref{5}

    \begin{equation}
      \frac{\lim_{h\to 0}f(x+h)+\lim_{h\to 0}f(x-h)}{2}
      \label{5}
    \end{equation}

\end{itemize}

\end{document}

