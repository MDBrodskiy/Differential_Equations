%%%%%%%%%%%%%%%%%%%%%%%%%%%%%%%%%%%%%%%%%%%%%%%%%%%%%%%%%%%%%%%%%%%%%%%%%%%%%%%%%%%%%%%%%%%%%%%%%%%%%%%%%%%%%%%%%%%%%%%%%%%%%%%%%%%%%%%%%%%%%%%%%%%%%%%%%%%%%%%%%%%%%%%%%%%%%%%%%%%%%%%%%%%%
% Written By Michael Brodskiy
% Class: Differential Equations (MATH-294)
% Professor: M. Shah
%%%%%%%%%%%%%%%%%%%%%%%%%%%%%%%%%%%%%%%%%%%%%%%%%%%%%%%%%%%%%%%%%%%%%%%%%%%%%%%%%%%%%%%%%%%%%%%%%%%%%%%%%%%%%%%%%%%%%%%%%%%%%%%%%%%%%%%%%%%%%%%%%%%%%%%%%%%%%%%%%%%%%%%%%%%%%%%%%%%%%%%%%%%%

\documentclass[12pt]{article} 
\usepackage{alphalph}
\usepackage[utf8]{inputenc}
\usepackage[russian,english]{babel}
\usepackage{titling}
\usepackage{amsmath}
\usepackage{graphicx}
\usepackage{enumitem}
\usepackage{amssymb}
\usepackage[super]{nth}
\usepackage{everysel}
\usepackage{ragged2e}
\usepackage{geometry}
\usepackage{fancyhdr}
\usepackage{cancel}
\usepackage{siunitx}
\geometry{top=1.0in,bottom=1.0in,left=1.0in,right=1.0in}
\newcommand{\subtitle}[1]{%
  \posttitle{%
    \par\end{center}
    \begin{center}\large#1\end{center}
    \vskip0.5em}%

}
\usepackage{hyperref}
\hypersetup{
colorlinks=true,
linkcolor=blue,
filecolor=magenta,      
urlcolor=blue,
citecolor=blue,
}

\urlstyle{same}


\title{Homogeneous Linear Equations with Constant Coefficients}
\date{\today}
\author{Michael Brodskiy\\ \small Professor: Meetal Shah}

% Mathematical Operations:

% Sum: $$\sum_{n=a}^{b} f(x) $$
% Integral: $$\int_{lower}^{upper} f(x) dx$$
% Limit: $$\lim_{x\to\infty} f(x)$$

\begin{document}

\maketitle

\begin{itemize}

  \item Given $ay'+by=0$, we can solve the differential equation by using something close to a guess by using $y=e^{mx}$ \eqref{1}

    \begin{equation}
      \begin{split}
        ay'+by=0\\
        y=e^{mx},\,y'=me^{mx}\\
        e^{mx}(am+b)=0\\
        m=\frac{-b}{a}\\
        y=c_1e^{\frac{-b}{a}}
      \end{split}
      \label{1}
    \end{equation}

  \item For the differential equation $ay''+by'+cy=0$, we may use the following method \eqref{2}

    \begin{equation}
      \begin{split}
        ay''+by'+cy=0\\
        y=e^{mx},\,y'=me^{mx},\,y''=m^2e^{mx}\\
        e^{mx}(am^2+bm+c)=0\\
        am^2+bm+c=0
      \end{split}
      \label{2}
    \end{equation}

  \item If the quadratic obtained in \eqref{2} has no real (and therefore imaginary) solutions, Euler's formula is used \eqref{3}

    \begin{equation}
      e^{i\theta}=\cos \theta + i\sin \theta
      \label{3}
    \end{equation}

  \item Two important equations to know are \eqref{4}

    \begin{equation}
      \begin{split}
        y''+k^2y=0\\
        y''-k^2y=0\\
        m^2+k^2=0 &\,\,\,\,\, m^2-k^2=0\\
      \end{split}
      \label{4}
    \end{equation}

  \item For higher order equations \eqref{5} a same solution could be applied \eqref{6}

    \begin{equation}
      a_nm^n+a_{n-1}m^{n-1}+\dots+a_2m^2+a_1m+a_0=0
      \label{5}
    \end{equation}

    \begin{equation}
      y=c_1e^m_1x+c_2e^m_2x+\dots+c_ne^m_nx
      \label{6}
    \end{equation}



\end{itemize}

\end{document}

