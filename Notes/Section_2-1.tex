%%%%%%%%%%%%%%%%%%%%%%%%%%%%%%%%%%%%%%%%%%%%%%%%%%%%%%%%%%%%%%%%%%%%%%%%%%%%%%%%%%%%%%%%%%%%%%%%%%%%%%%%%%%%%%%%%%%%%%%%%%%%%%%%%%%%%%%%%%%%%%%%%%%%%%%%%%%%%%%%%%%%%%%%%%%%%%%%%%%%%%%%%%%%
% Written By Michael Brodskiy
% Class: Differential Equations (MATH-294)
% Professor: M. Shah
%%%%%%%%%%%%%%%%%%%%%%%%%%%%%%%%%%%%%%%%%%%%%%%%%%%%%%%%%%%%%%%%%%%%%%%%%%%%%%%%%%%%%%%%%%%%%%%%%%%%%%%%%%%%%%%%%%%%%%%%%%%%%%%%%%%%%%%%%%%%%%%%%%%%%%%%%%%%%%%%%%%%%%%%%%%%%%%%%%%%%%%%%%%%

\documentclass[12pt]{article} 
\usepackage{alphalph}
\usepackage[utf8]{inputenc}
\usepackage[russian,english]{babel}
\usepackage{titling}
\usepackage{amsmath}
\usepackage{graphicx}
\usepackage{enumitem}
\usepackage{amssymb}
\usepackage[super]{nth}
\usepackage{everysel}
\usepackage{ragged2e}
\usepackage{geometry}
\usepackage{fancyhdr}
\usepackage{cancel}
\usepackage{siunitx}
\geometry{top=1.0in,bottom=1.0in,left=1.0in,right=1.0in}
\newcommand{\subtitle}[1]{%
  \posttitle{%
    \par\end{center}
    \begin{center}\large#1\end{center}
    \vskip0.5em}%

}
\usepackage{hyperref}
\hypersetup{
colorlinks=true,
linkcolor=blue,
filecolor=magenta,      
urlcolor=blue,
citecolor=blue,
}

\urlstyle{same}


\title{Solution Curves Without a Solution}
\date{\today}
\author{Michael Brodskiy\\ \small Professor: Meetal Shah}

% Mathematical Operations:

% Sum: $$\sum_{n=a}^{b} f(x) $$
% Integral: $$\int_{lower}^{upper} f(x) dx$$
% Limit: $$\lim_{x\to\infty} f(x)$$

\begin{document}

\maketitle

\begin{itemize}

  \item The function $f$ in the normal form is called the slope function or rate function 

    $$\frac{dy}{dx}=f(x,y)$$

  \item Lineal Element $-$ An individual slope at a certain point

  \item Direction Field (Slope Field) $-$ The collection of all lineal elements

  \item Autonomous First-Order Differential Equation:

    $$\frac{dy}{dx}=f(y)$$

  \item Critical Points $-$ Also called equilibrium or stationary points, they are points, $c$, which, when plugged into a function, yield:

    $$f(c)=0$$

  \item Equilibrium Solution $-$ If $c$ is a critical point, then $y(x)=c$ is a constant solution of the autonomous differential equation. Equilibria are the only constant solutions

  \item One-Dimensional Phase Portrait $-$ Simply called a phase portrait, it shows the intervals where a function is increasing and decreasing (essentially a sign chart). The line that the intervals are graphed on is named the phase line

  \item Attractors and Repellers $-$ For a nonconstant solution, $y(x)$, there are basically three behaviors as it approaches some critical point, $c$. 
    
    \begin{enumerate}

      \item Attractor $-$ When $\lim_{x\to\pm\infty}y(x)=c$. This means the critical point is asymptotically stable.

      \item Repeller $-$ When both parts of a solution move away from a $c$ point. These are unstable.

      \item Semi-stable $-$ When both parts either move up or down.

    \end{enumerate}

\end{itemize}

\end{document}

