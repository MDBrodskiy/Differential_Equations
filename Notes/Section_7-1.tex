%%%%%%%%%%%%%%%%%%%%%%%%%%%%%%%%%%%%%%%%%%%%%%%%%%%%%%%%%%%%%%%%%%%%%%%%%%%%%%%%%%%%%%%%%%%%%%%%%%%%%%%%%%%%%%%%%%%%%%%%%%%%%%%%%%%%%%%%%%%%%%%%%%%%%%%%%%%%%%%%%%%%%%%%%%%%%%%%%%%%%%%%%%%%
% Written By Michael Brodskiy
% Class: Differential Equations (MATH-294)
% Professor: M. Shah
%%%%%%%%%%%%%%%%%%%%%%%%%%%%%%%%%%%%%%%%%%%%%%%%%%%%%%%%%%%%%%%%%%%%%%%%%%%%%%%%%%%%%%%%%%%%%%%%%%%%%%%%%%%%%%%%%%%%%%%%%%%%%%%%%%%%%%%%%%%%%%%%%%%%%%%%%%%%%%%%%%%%%%%%%%%%%%%%%%%%%%%%%%%%

\documentclass[12pt]{article} 
\usepackage{alphalph}
\usepackage[utf8]{inputenc}
\usepackage[russian,english]{babel}
\usepackage{titling}
\usepackage{amsmath}
\usepackage{graphicx}
\usepackage{enumitem}
\usepackage{amssymb}
\usepackage[super]{nth}
\usepackage{everysel}
\usepackage{ragged2e}
\usepackage{geometry}
\usepackage{fancyhdr}
\usepackage{cancel}
\usepackage{siunitx}
\geometry{top=1.0in,bottom=1.0in,left=1.0in,right=1.0in}
\newcommand{\subtitle}[1]{%
  \posttitle{%
    \par\end{center}
    \begin{center}\large#1\end{center}
    \vskip0.5em}%

}
\usepackage{hyperref}
\hypersetup{
colorlinks=true,
linkcolor=blue,
filecolor=magenta,      
urlcolor=blue,
citecolor=blue,
}

\urlstyle{same}


\title{Definition of the Laplace Transform}
\date{\today}
\author{Michael Brodskiy\\ \small Professor: Meetal Shah}

% Mathematical Operations:

% Sum: $$\sum_{n=a}^{b} f(x) $$
% Integral: $$\int_{lower}^{upper} f(x) dx$$
% Limit: $$\lim_{x\to\infty} f(x)$$

\begin{document}

\maketitle

\begin{itemize}

  \item Derivatives and integrals follow the linearity property, or, for constants $\alpha$ and $\beta$:

    \begin{equation}
      \begin{split}
        \frac{d}{dx}[\alpha f(x) + \beta g(x)]&=\alpha f'(x) + \beta g'(x)\\
        \int[\alpha f(x) + \beta g(x)]\,dx&=\alpha \int f(x)\,dx+\beta\int g(x)\,dx
      \end{split}
      \label{1}
    \end{equation}

  \item Integral transforms are done as such:

    \begin{equation}
      \int_0^{\infty} K(s,t)f(t)\,dt=\lim_{b\to\infty}\int_0^b K(s,t)f(t)\,dt
      \label{2}
    \end{equation}

  \item The Laplace Transform is defined as, where $K(s,t)$ is the kernel of the transform:

    \begin{equation}
      \mathcal{L} =\int_0^{\infty}e^{-st}f(t)\,dt\\
      \label{3}
    \end{equation}

  \item Then, for two different functions, the Laplace transform would be \eqref{4}, which means it is a linear transform

    \begin{equation}
      \mathcal{L}\{\alpha f(x) + \beta g(x)\}=\alpha \int e^{-st}f(x)\,dx+\beta\int e^{-st}g(x)\,dx
      \label{4}
    \end{equation}

  \item Laplace Transforms for common functions:

    \begin{equation}
      \begin{split}
        \mathcal{L}\{1\}&=\frac{1}{s}\\
        \mathcal{L}\{t^n\}=\frac{n!}{s^{n+1}}\,\,\,\,&\,\,\,\, \mathcal{L}\{e^{at}\}=\frac{1}{s-a}\\
        \mathcal{L}\{\sin kt\}=\frac{k}{s^2+k^2}\,\,\,\,&\,\,\,\, \mathcal{L}\{\cos kt\}=\frac{s}{s^2+k^2}\\
        \mathcal{L}\{\sinh kt\}=\frac{k}{s^2-k^2}\,\,\,\,&\,\,\,\, \mathcal{L}\{\cosh kt\}=\frac{s}{s^2-k^2}\\
      \end{split}
      \label{5}
    \end{equation}

  \item For a Laplace Transform to exist, the function presented must be piecewise continuous and of exponential order

  \item A function $f$ is said to be of exponential order if there exist constants, $c$, $M>0$, and $T>0$ such that $|f(t)|\leq Me^{ct}$ for all $t>T$

  \item If $f$ is piecewise continuous on $[0,\infty)$ and of exponential order, then $\mathcal{L}\{f(t)\}$ exists for $s>c$

  \item If $f$ is piecewise continuous on $[0,\infty)$ and of exponential order and $F(s)=\mathcal{L}\{f(t)\}$, then $\lim_{s\to\infty}F(s)=0$

\end{itemize}

\end{document}

