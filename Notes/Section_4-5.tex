%%%%%%%%%%%%%%%%%%%%%%%%%%%%%%%%%%%%%%%%%%%%%%%%%%%%%%%%%%%%%%%%%%%%%%%%%%%%%%%%%%%%%%%%%%%%%%%%%%%%%%%%%%%%%%%%%%%%%%%%%%%%%%%%%%%%%%%%%%%%%%%%%%%%%%%%%%%%%%%%%%%%%%%%%%%%%%%%%%%%%%%%%%%%
% Written By Michael Brodskiy
% Class: Differential Equations (MATH-294)
% Professor: M. Shah
%%%%%%%%%%%%%%%%%%%%%%%%%%%%%%%%%%%%%%%%%%%%%%%%%%%%%%%%%%%%%%%%%%%%%%%%%%%%%%%%%%%%%%%%%%%%%%%%%%%%%%%%%%%%%%%%%%%%%%%%%%%%%%%%%%%%%%%%%%%%%%%%%%%%%%%%%%%%%%%%%%%%%%%%%%%%%%%%%%%%%%%%%%%%

\documentclass[12pt]{article} 
\usepackage{alphalph}
\usepackage[utf8]{inputenc}
\usepackage[russian,english]{babel}
\usepackage{titling}
\usepackage{amsmath}
\usepackage{graphicx}
\usepackage{enumitem}
\usepackage{amssymb}
\usepackage[super]{nth}
\usepackage{everysel}
\usepackage{ragged2e}
\usepackage{geometry}
\usepackage{fancyhdr}
\usepackage{cancel}
\usepackage{siunitx}
\geometry{top=1.0in,bottom=1.0in,left=1.0in,right=1.0in}
\newcommand{\subtitle}[1]{%
  \posttitle{%
    \par\end{center}
    \begin{center}\large#1\end{center}
    \vskip0.5em}%

}
\usepackage{hyperref}
\hypersetup{
colorlinks=true,
linkcolor=blue,
filecolor=magenta,      
urlcolor=blue,
citecolor=blue,
}

\urlstyle{same}


\title{Undetermined Coefficients $-$ Annihilator Approach}
\date{\today}
\author{Michael Brodskiy\\ \small Professor: Meetal Shah}

% Mathematical Operations:

% Sum: $$\sum_{n=a}^{b} f(x) $$
% Integral: $$\int_{lower}^{upper} f(x) dx$$
% Limit: $$\lim_{x\to\infty} f(x)$$

\begin{document}

\maketitle

\begin{itemize}

  \item An $n$th-order differential equation can be written in form \eqref{1}, where $D^ky=d^ky/dx^k$ 

    \begin{equation}
      a_nD^ny+a_{n-1}D^{n-1}y+\dots+a_1Dy+a_0y=g(x)
      \label{1}
    \end{equation}

  \item Operators may be factored just like polynomial equations: \eqref{2}

    \begin{equation}
      (D^2+5D+6)y = (D+3)(D+2)y
      \label{2}
    \end{equation}

  \item The linear operator $L$ is said to be an \textbf{annihilator} if it has the property shown in \eqref{3}

    \begin{equation}
      L(f(x))=0
      \label{3}
    \end{equation}

    \begin{itemize}

      \item For example:

      \item $D$ annihilates $y=k$ because $Dk=0$

      \item $D^2$ annihilates $y=x$ because $D^2x=0$

      \item $D^3$ annihilates $y=x^2$ because $D^3x^2=0$

    \end{itemize}

  \item Some common annihilators are listed:

    \begin{equation}
      \begin{split}
        D^n \text{ annihilates } & x^{n-1}\\
        (D-\alpha)^n \text{ annihilates } & x^{n-1}e^{\alpha x}\\
        [D^2-2\alpha D+(\alpha^2+\beta^2)]^n \text{ annihilates } & x^{n-1}e^{\alpha x} \cos \beta x,\, x^{n-1}e^{\alpha x} \sin \beta x
      \end{split}
      \label{4}
    \end{equation}

  \item Once the annihilator is solved for, the differential equation becomes homogeneous and can be easily solved

\end{itemize}

\end{document}

