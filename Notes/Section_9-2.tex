%%%%%%%%%%%%%%%%%%%%%%%%%%%%%%%%%%%%%%%%%%%%%%%%%%%%%%%%%%%%%%%%%%%%%%%%%%%%%%%%%%%%%%%%%%%%%%%%%%%%%%%%%%%%%%%%%%%%%%%%%%%%%%%%%%%%%%%%%%%%%%%%%%%%%%%%%%%%%%%%%%%%%%%%%%%%%%%%%%%%%%%%%%%%
% Written By Michael Brodskiy
% Class: Differential Equations (MATH-294)
% Professor: M. Shah
%%%%%%%%%%%%%%%%%%%%%%%%%%%%%%%%%%%%%%%%%%%%%%%%%%%%%%%%%%%%%%%%%%%%%%%%%%%%%%%%%%%%%%%%%%%%%%%%%%%%%%%%%%%%%%%%%%%%%%%%%%%%%%%%%%%%%%%%%%%%%%%%%%%%%%%%%%%%%%%%%%%%%%%%%%%%%%%%%%%%%%%%%%%%

\documentclass[12pt]{article} 
\usepackage{alphalph}
\usepackage[utf8]{inputenc}
\usepackage[russian,english]{babel}
\usepackage{titling}
\usepackage{amsmath}
\usepackage{graphicx}
\usepackage{enumitem}
\usepackage{amssymb}
\usepackage[super]{nth}
\usepackage{everysel}
\usepackage{ragged2e}
\usepackage{geometry}
\usepackage{fancyhdr}
\usepackage{cancel}
\usepackage{siunitx}
\geometry{top=1.0in,bottom=1.0in,left=1.0in,right=1.0in}
\newcommand{\subtitle}[1]{%
  \posttitle{%
    \par\end{center}
    \begin{center}\large#1\end{center}
    \vskip0.5em}%

}
\usepackage{hyperref}
\hypersetup{
colorlinks=true,
linkcolor=blue,
filecolor=magenta,      
urlcolor=blue,
citecolor=blue,
}

\urlstyle{same}


\title{Runge-Kutta Methods}
\date{\today}
\author{Michael Brodskiy\\ \small Professor: Meetal Shah}

% Mathematical Operations:

% Sum: $$\sum_{n=a}^{b} f(x) $$
% Integral: $$\int_{lower}^{upper} f(x) dx$$
% Limit: $$\lim_{x\to\infty} f(x)$$

\begin{document}

\maketitle

\begin{itemize}

  \item Much like in probability, the sum of the weights, $w_1+w_2+\dots+w_m=1$. The order of the Runge-Kutta Method is given by $m$, or the amount of weights given. Therefore, a first-order Runge-Kutta Method is Euler's Method.

  \item To find the fourth-order Runge-Kutta procedure, one must use formula \eqref{1}

    \begin{equation}
      \begin{split}
        y_{n+1}&=y_n+h(w_1k_1+w_2k_2+w_3k_3+w_4k_4)\\
        \text{Where}&\\
        k_1&=f(x_n,y_n)\\
        k_2&=f(x_n+\alpha_1h,y_n+\beta_1hk_1)\\
        k_3&=f(x_n+\alpha_2h,y_n+\beta_2hk_1+\beta_3hk_2)\\
        k_4&=f(x_n+\alpha_3h,y_n+\beta_4hk_1+\beta_5hk_2+\beta_6hk_3)\\
      \end{split}
      \label{1}
    \end{equation}

  \item For our purposes, we go with the formula \eqref{2}

    \begin{equation}
      \begin{split}
        y_{n+1}&=y_n+\frac{h}{6}(k_1+2k_2+2k_3+k_4)\\
        k_1&=f(x_n,y_n)\\
        k_2&=f(x_n+\frac{1}{2}h,y_n+\frac{1}{2}hk_1)\\
        k_3&=f(x_n+\frac{1}{2}h,y_n+\frac{1}{2}hk_2)\\
        k_4&=f(x_n+h,y_n+hk_3)\\
      \end{split}
      \label{2}
    \end{equation}

  \item \eqref{2} is known as \underline{the} \textbf{RK4 method} 

  \item Because RK4 is of order 4, the local truncation error is $O(h^5)$ and the global truncation error is $O(h^4)$

  \item More accurate methods, called adaptive methods do exist. One such method is the RKF45 method, or the Runge-Kutta-Fehlberg method. 

\end{itemize}

\end{document}

