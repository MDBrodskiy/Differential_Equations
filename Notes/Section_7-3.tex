%%%%%%%%%%%%%%%%%%%%%%%%%%%%%%%%%%%%%%%%%%%%%%%%%%%%%%%%%%%%%%%%%%%%%%%%%%%%%%%%%%%%%%%%%%%%%%%%%%%%%%%%%%%%%%%%%%%%%%%%%%%%%%%%%%%%%%%%%%%%%%%%%%%%%%%%%%%%%%%%%%%%%%%%%%%%%%%%%%%%%%%%%%%%
% Written By Michael Brodskiy
% Class: Differential Equations (MATH-294)
% Professor: M. Shah
%%%%%%%%%%%%%%%%%%%%%%%%%%%%%%%%%%%%%%%%%%%%%%%%%%%%%%%%%%%%%%%%%%%%%%%%%%%%%%%%%%%%%%%%%%%%%%%%%%%%%%%%%%%%%%%%%%%%%%%%%%%%%%%%%%%%%%%%%%%%%%%%%%%%%%%%%%%%%%%%%%%%%%%%%%%%%%%%%%%%%%%%%%%%

\documentclass[12pt]{article} 
\usepackage{alphalph}
\usepackage[utf8]{inputenc}
\usepackage[russian,english]{babel}
\usepackage{titling}
\usepackage{amsmath}
\usepackage{graphicx}
\usepackage{enumitem}
\usepackage{amssymb}
\usepackage[super]{nth}
\usepackage{everysel}
\usepackage{ragged2e}
\usepackage{geometry}
\usepackage{fancyhdr}
\usepackage{cancel}
\usepackage{siunitx}
\geometry{top=1.0in,bottom=1.0in,left=1.0in,right=1.0in}
\newcommand{\subtitle}[1]{%
  \posttitle{%
    \par\end{center}
    \begin{center}\large#1\end{center}
    \vskip0.5em}%

}
\usepackage{hyperref}
\hypersetup{
colorlinks=true,
linkcolor=blue,
filecolor=magenta,      
urlcolor=blue,
citecolor=blue,
}

\urlstyle{same}


\title{Operational Properties}
\date{\today}
\author{Michael Brodskiy\\ \small Professor: Meetal Shah}

% Mathematical Operations:

% Sum: $$\sum_{n=a}^{b} f(x) $$
% Integral: $$\int_{lower}^{upper} f(x) dx$$
% Limit: $$\lim_{x\to\infty} f(x)$$

\begin{document}

\maketitle

\begin{itemize}

  \item If $\mathcal{L}\{f(t)\}=F(s)$ and $a$ is any real number, then:

    \begin{equation}
      \mathcal{L}\{e^{at}f(t)\}=F(s-a)
      \label{1}
    \end{equation}

  \item This theorem works both ways, as:

    \begin{equation}
      e^{at}f(t)=\mathcal{L}^{-1}\{F(s-a)\}
      \label{2}
    \end{equation}

    \textit{Example:}

    \begin{equation}
      \begin{split}
        \mathcal{L}^{-1}\left\{ \frac{\frac{s}{2}+\frac{5}{3}}{s^2+4s+6} \right\}\\
        =\frac{1}{2}\left( \frac{s+2}{(s+2)^2+2} \right)+\frac{2}{3}\left( \frac{1}{(s+2)^2+2} \right)\\
        s\to s+2=\frac{1}{2}\mathcal{L}^{-1}\left\{ \frac{s}{s^2+2} \right\}+\frac{2}{3\sqrt{2}}\mathcal{L}^{-1}\left\{ \frac{\sqrt{2}}{s^2+2}  \right\}\\
          =\frac{1}{2}e^{-2t}\cos \sqrt{2}t+\frac{\sqrt{2}}{3}e^{-2t}\sin\sqrt{2}t
      \end{split}
      \label{3}
    \end{equation}

  \item The unit step function, $\mathcal{U}(t-a)=\left\{\begin{array}{ll} 0 & 0\leq t< a\\ 1 & t\geq a \end{array}$, may be used to represent functions.

    \item For example, given function: $f(t)=\left\{\begin{array}{ll} g(t) & 0\leq t\leq a\\ h(t) & t\geq a \end{array}$, one may represent it using step functions in the way expressed in \eqref{4}. As such, the function is $h(t)$ when $t\geq a$, and $g(t)$ when $0\leq t< a$.

        \begin{equation}
          f(t)=g(t)-g(t)\mathcal{U}(t-a)+h(t)\mathcal{U}(t-a)
          \label{4}
        \end{equation}

      \item The Second Translation Theorem holds that: $\mathcal{L}\left\{ f(t-a)\mathcal{U}(t-a) \right\}=e^{-as}F(s)$

      \item The Inverse of the Second Translation Theorem holds that: $f(t-a)\mathcal{U}(t-a)=\mathcal{L}^{-1}\left\{e^{-as}F(s)\right\}$

      \item A better way of figuring out $\mathcal{L}\left\{ g(t)\mathcal{U}(t-a) \right\}$ is the following:

        \begin{equation}
          \begin{split}
          \int_a^{\infty} e^{-st}g(t)\,dt=\int_0^{\infty} e^{-s(u+a)}g(u+a)\,du\\
          =e^{-as}\mathcal{L}\left\{ g(t+a)  \right\}
        \end{split}
          \label{5}
        \end{equation}

      \item Now, the Laplace Transform may be applied to one of the previous sections, where $EI\frac{d^4y}{dx^4}=w(x)$, where $E$ is Young's modulus of elasticity and $I$ is a moment of inertia of a cross section of the beam
        
      \item Given $w(x)=\left\{\begin{array}{ll} w_0\left( 1-\frac{2}{L}x \right) & 0<x<\frac{L}{2} \\ 0 & \frac{L}{2} < x < L \end{array}$ means that the beam is embedded at both ends, with force applied only to the left, meaning that $y(0)=0$, $y'(0)=0$, $y(L)=0$, and $y'(L)=0$

\end{itemize}

\end{document}

