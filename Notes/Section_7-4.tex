%%%%%%%%%%%%%%%%%%%%%%%%%%%%%%%%%%%%%%%%%%%%%%%%%%%%%%%%%%%%%%%%%%%%%%%%%%%%%%%%%%%%%%%%%%%%%%%%%%%%%%%%%%%%%%%%%%%%%%%%%%%%%%%%%%%%%%%%%%%%%%%%%%%%%%%%%%%%%%%%%%%%%%%%%%%%%%%%%%%%%%%%%%%%
% Written By Michael Brodskiy
% Class: Differential Equations (MATH-294)
% Professor: M. Shah
%%%%%%%%%%%%%%%%%%%%%%%%%%%%%%%%%%%%%%%%%%%%%%%%%%%%%%%%%%%%%%%%%%%%%%%%%%%%%%%%%%%%%%%%%%%%%%%%%%%%%%%%%%%%%%%%%%%%%%%%%%%%%%%%%%%%%%%%%%%%%%%%%%%%%%%%%%%%%%%%%%%%%%%%%%%%%%%%%%%%%%%%%%%%

\documentclass[12pt]{article} 
\usepackage{alphalph}
\usepackage[utf8]{inputenc}
\usepackage[russian,english]{babel}
\usepackage{titling}
\usepackage{amsmath}
\usepackage{graphicx}
\usepackage{enumitem}
\usepackage{amssymb}
\usepackage[super]{nth}
\usepackage{everysel}
\usepackage{ragged2e}
\usepackage{geometry}
\usepackage{fancyhdr}
\usepackage{cancel}
\usepackage{siunitx}
\geometry{top=1.0in,bottom=1.0in,left=1.0in,right=1.0in}
\newcommand{\subtitle}[1]{%
  \posttitle{%
    \par\end{center}
    \begin{center}\large#1\end{center}
    \vskip0.5em}%

}
\usepackage{hyperref}
\hypersetup{
colorlinks=true,
linkcolor=blue,
filecolor=magenta,      
urlcolor=blue,
citecolor=blue,
}

\urlstyle{same}


\title{Operational Properties II}
\date{\today}
\author{Michael Brodskiy\\ \small Professor: Meetal Shah}

% Mathematical Operations:

% Sum: $$\sum_{n=a}^{b} f(x) $$
% Integral: $$\int_{lower}^{upper} f(x) dx$$
% Limit: $$\lim_{x\to\infty} f(x)$$

\begin{document}

\maketitle

\begin{itemize}

  \item The derivative of a transform is: $\mathcal{L}\left\{ t^nf(t) \right\}=(-1)^n\frac{d^n}{ds^n}F(s)$

  \item A convolution (f convolves g) is defined as \eqref{1}

    \begin{equation}
      f\ast g=\int_0^t f(\tau)g(t-\tau)\,d\tau
      \label{1}
    \end{equation}

  \item For all examples, the Laplace transform of a convolution, or $\mathcal{L}\{f\ast g\}$, we obtain \eqref{2}

    \begin{equation}
      \begin{split}
        \mathcal{L}\left\{ f\ast g \right\}=\mathcal{L}\left\{ f(t) \right\}\cdot\mathcal{L}\left\{ g(t) \right\}=F(s)G(s)
      \end{split}
      \label{2}
    \end{equation}

  \item Inversely, the inverse Laplace transform would mean \eqref{3}

    \begin{equation}
      f\ast g=\mathcal{L}^{-1}{F(s)G(s)}
      \label{3}
    \end{equation}
  
  \item When $g(t)=1$ and $\mathcal{L}\left\{ g(t) \right\}=G(s)=\frac{1}{s}$, the convolution theorem implies that the Laplace transform of the integral of $f$ is \eqref{4}

    \begin{equation}
      \begin{split}
      \mathcal{L}\left\{ \int_0^t f(\tau)\,d\tau \right\}=\frac{F(s)}{s}\\
      \int_0^t f(\tau)\,d\tau=\mathcal{L}^{-1}\left\{\frac{F(s)}{s}\right\}\\
    \end{split}
      \label{4}
    \end{equation}

  \item The Volterra integral equation for $f(t)$ is \eqref{5}

    \begin{equation}
      f(t)=g(t)+\int_0^t f(\tau)h(t-\tau)\,d\tau
      \label{5}
    \end{equation}

  \item For LRC circuits, we could set up an integral equation to solve. This would look like \eqref{6}, which is known as an integrodifferential equation

    \begin{equation}
      L\frac{di}{dt}+Ri(t)+\frac{1}{C}\int_0^ti(\tau)\,d\tau=E(t)
      \label{6}
    \end{equation}

  \item Green's Function Redux states that, given a differential equation, $y''+ay'+by=f(t)$, with $y(0)=0, y'(0)=0$, we attain \eqref{7}

    \begin{equation}
      Y(s)=\frac{F(s)}{s^2+as+b}
      \label{7}
    \end{equation}

  \item From there, we can use convolution, as $\mathcal{L}^{-1}\left\{ \frac{1}{s^2+as+b} \right\}=g(t)$ and $\mathcal{L}^{-1}\left\{ F(s) \right\}=f(t)$

  \item The transform of a periodic function becomes \eqref{8}

    \begin{equation}
      \mathcal{L}\left\{ f(t) \right\}=\frac{1}{1-e^{-sT}}\int_0^T e^{-st}f(t)\,dt
      \label{8}
    \end{equation}

    \textit{Example:}

    \begin{center}
      Given $E(t)=\left\{\begin{array}{ll} 1 & 0\leq t< 1\\ 0 & 1\leq t <2 \end{array}$, find $\mathcal{L}\left\{ E(t) \right\}$
    \end{center}

    \begin{equation}
      \mathcal{L}\left\{ E(t) \right\}=\frac{1}{1-e^{-2s}}\int_0^2e^{-st}E(t)\,dt
      \label{9}
    \end{equation}

\end{itemize}

\end{document}

