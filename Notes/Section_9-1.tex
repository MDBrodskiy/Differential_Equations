%%%%%%%%%%%%%%%%%%%%%%%%%%%%%%%%%%%%%%%%%%%%%%%%%%%%%%%%%%%%%%%%%%%%%%%%%%%%%%%%%%%%%%%%%%%%%%%%%%%%%%%%%%%%%%%%%%%%%%%%%%%%%%%%%%%%%%%%%%%%%%%%%%%%%%%%%%%%%%%%%%%%%%%%%%%%%%%%%%%%%%%%%%%%
% Written By Michael Brodskiy
% Class: Differential Equations (MATH-294)
% Professor: M. Shah
%%%%%%%%%%%%%%%%%%%%%%%%%%%%%%%%%%%%%%%%%%%%%%%%%%%%%%%%%%%%%%%%%%%%%%%%%%%%%%%%%%%%%%%%%%%%%%%%%%%%%%%%%%%%%%%%%%%%%%%%%%%%%%%%%%%%%%%%%%%%%%%%%%%%%%%%%%%%%%%%%%%%%%%%%%%%%%%%%%%%%%%%%%%%

\documentclass[12pt]{article} 
\usepackage{alphalph}
\usepackage[utf8]{inputenc}
\usepackage[russian,english]{babel}
\usepackage{titling}
\usepackage{amsmath}
\usepackage{graphicx}
\usepackage{enumitem}
\usepackage{amssymb}
\usepackage[super]{nth}
\usepackage{everysel}
\usepackage{ragged2e}
\usepackage{geometry}
\usepackage{fancyhdr}
\usepackage{cancel}
\usepackage{siunitx}
\geometry{top=1.0in,bottom=1.0in,left=1.0in,right=1.0in}
\newcommand{\subtitle}[1]{%
  \posttitle{%
    \par\end{center}
    \begin{center}\large#1\end{center}
    \vskip0.5em}%

}
\usepackage{hyperref}
\hypersetup{
colorlinks=true,
linkcolor=blue,
filecolor=magenta,      
urlcolor=blue,
citecolor=blue,
}

\urlstyle{same}


\title{Euler Methods and Error Analysis}
\date{\today}
\author{Michael Brodskiy\\ \small Professor: Meetal Shah}

% Mathematical Operations:

% Sum: $$\sum_{n=a}^{b} f(x) $$
% Integral: $$\int_{lower}^{upper} f(x) dx$$
% Limit: $$\lim_{x\to\infty} f(x)$$

\begin{document}

\maketitle

\begin{itemize}

  \item The backbone of Euler's Method is formula \eqref{1}

    \begin{equation}
      y_{n+1}=y_n+hf(x_n,y_n)
      \label{1}
    \end{equation}

  \item Round-off Error $-$ This error occurs when an object doing the calculating has an error due to the finite number of digits it displays.

  \item Truncation Error $-$ An error that occurs at each step, and, therefore, carries over to the next step, until the calculator reaches a (total) truncation error.

    \begin{itemize}

      \item At a certain step, it is called \underline{local} truncation error, while, over the whole scope, it is called \under{global} truncation error.

    \end{itemize}

  \item When using Euler's method, the local truncation error can be found by using the formula for a Taylor's Series \eqref{2}, where $\frac{h^2}{2!}$ is known as the local truncation error.

    \begin{equation}
      y(x_{n+1})=y_n+hf(x_n,y_n)+y''(c)\frac{h^2}{2!}
      \label{2}
    \end{equation}

  \item If $e(h)$ denotes the error in a numerical calculation depending on $h$ then $e(h)$ is said to be of order $h^n$, denoted by $O(h^n)$, if there exists a constant $C$ and a positive integer $n$ such that $|e(h)|\leq Ch^n$ for $h$ sufficiently small.

  \item The improved Euler's Method is defined as \eqref{3}

    \begin{equation}
      \begin{split}
      y_{n+1}=y_n+h\frac{f(x_n,y_n)+f(x_{n+1},y^*_{n-1})}{2}\\
      y^*_{n+1}=y_n+hf(x_n,y_n)
    \end{split}
      \label{3}
    \end{equation}

  \item The improved Euler's Method is an example of a predictor-corrector method. The value of $y^*_{n+1}$ given by \eqref{3} predicts a value of $y(x_n)$, whereas the value of $y_{n+1}$ defined by formula \eqref{3} corrects this estimate. 

  \item The local truncation error for the improved Euler's method is $O(h^3)$. The global truncation error, therefore, is $O(h^2)$

\end{itemize}

\end{document}

